%----------------------------------------------------------------------------
\chapter{Conclusion}\label{ch:conclusion}
%----------------------------------------------------------------------------


The main goal of this semester's work was to investigate the legitimacy of a number of different algorithms in the field of \textit{de novo} molecule design for reducing the dimension of a dataset in a meaningful way, allowing for performing targeted searching. I have learned much about each algorithm, and can confidently say which ones are particularly useful for this use-case and which ones are less so.

I have found that -- as many experts already know -- PCA is fundamentally unfit for such a task, as it has severe limitations in the correlations it can capture. t-SNE, which to this day is one of if not the most popular dimension reduction algorithm suffers from near-sightedness and performance issues both in terms of memory usage and runtime (the later of which was not an issue because of the GPU implementation). UMAP seems to be among the most useful algorithms in this regard, transforming the chemical space in such a way as to be able to smoothly transition from one molecule to the next. PaCMAP showed the most potential in the tests as the \textit{de facto}  go-to algorithm for pharmaceutical research. Unfortunately, the potential of TriMAP was not explored well enough for any definitive statement, however, educated guesses can be made to suggest that it is suboptimal for allowing targeted searching on its output space.

In the future, I would like to dive deeper into the optimal parametrization of each method. With more parameters to test, I believe a more desirable output space can be achieved with each method. 

I would particularly like to test TriMAP more with the right parameters, because no conclusion can be made from the parameter optimization I performed. 

While the LERP test was an important and fruitful examination of each algorithm's embedding quality, an arguably more important property of the implementation is the possibility of inverse transformation. That is, by sampling points from the output space, can a model return their respective positions in the original space. This would make targeted searching even more powerful, as one can simply search in the output space -- an operation that is faster because the lowered dimensionality -- and produce the corresponding molecules by inverse transforming the sampled dataset.

Overall, the work I have done this semester answered quite a few questions about the validity of such dimension reduction algorithms in \textit{de novo} molecule design, but there are still more to investigate. 
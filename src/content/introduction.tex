%----------------------------------------------------------------------------
\chapter{\bevezetes}\label{ch:introduction}
%----------------------------------------------------------------------------

\section{Motivation}\label{sec:motivation}

Computational bio- and chemoinformatics often requires the usage and processing of high-dimensional data. Working with such data poses certain challenges however. While some datasets are constructed with meaningful and easily interpretable feature sets, often times the dimensions carry no direct interpretation, forming a \textit{latent space} of data points.

High dimensionality introduces redundancy in the representation of data points. It is also undesirable as the time of operations scale with the number of dimensions in the dataset.

Latent representation of data is advantageous for eliminating redundancy in the dataset. This is because features taken from real descriptors (such as number of carbon atoms in a molecule and molar mass) are often not independent of each other. These correlations of dimensions lead to spaces of lower dimension with near equivalent descriptive power to the original space. 

\section{Exploring the chemical space}\label{sec:exploring-the-chemical-space}


\section{Unsupervised clustering}\label{sec:unsupervised-clustering}

In data science, a very common task is discovering similarities and dissimilarities between data points. Categorizing points of data is perhaps the most obvious example of this, however, more general tasks exist in datasets where no such clean categories exist. These problems require a more general solution for representing similarities of points.

Cluster analysis is the grouping of objects such that objects in the same cluster are more similar to each other than they are to objects in another cluster. The classification into clusters is done using criteria such as smallest distances, density of data points, graphs, or various statistical distributions. Cluster analysis has  wide applicability, including in unsupervised machine learning, data mining, statistics, graph analytics,  image processing, and numerous physical and social science applications.

Clustering is used to identify groups of similar objects in datasets with two or more variable quantities. In practice, this data may be collected from marketing, biomedical, or geospatial databases, among datasets that come from preprocessing such databases.

Clustering is often used to discover underlying structure in data through \textit{unsupervised learning}. No labels are given to the learning algorithm, leaving it on its own to find structure in its input. Unsupervised learning can be a goal in itself (discovering hidden patterns in data) or a means towards an end (feature learning). 

\section{title}
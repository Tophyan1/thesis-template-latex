%----------------------------------------------------------------------------
\chapter{\bevezetes}
%----------------------------------------------------------------------------

\section{Motivation}

Computational bio- and chemoinformatics often requires the usage and processing of high-dimensional data. Working with such data poses certain challenges however. While some datasets are constructed with meaningful and easily interpretable feature sets, often times the dimensions carry no direct interpretation, forming a \textit{latent space} of data points.

High dimensionality introduces redundancy in the representation of data points. It is also undesirable as the time of operations scale with the number of dimensions in the dataset.

Latent representation of data is advantageous for eliminating redundancy in the dataset. This is because features taken from real descriptors (such as number of carbon atoms in a molecule and molar mass) are often not independent of each other. These correlations of dimensions lead to spaces of lower dimension with near equivalent descriptive power to the original space. 

\section{Exploring the chemical space}



\section{Unsupervised clustering}
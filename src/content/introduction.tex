%----------------------------------------------------------------------------
\chapter{\bevezetes}\label{ch:introduction}
%----------------------------------------------------------------------------

\section{Motivation}\label{sec:motivation}

Computational bio- and chemoinformatics often requires the usage and processing of high-dimensional data. Working with such data poses certain challenges however. While some datasets are constructed with meaningful and easily interpretable feature sets, often times the dimensions carry no direct interpretation, forming a \textit{latent space} of data points.

High dimensionality introduces redundancy in the representation of data points. It is also undesirable as the time of operations scale with the number of dimensions in the dataset.

Latent representation of data is advantageous for eliminating redundancy in the dataset. This is because features taken from real descriptors (such as number of carbon atoms in a molecule and molar mass) are often not independent of each other. These correlations of dimensions lead to spaces of lower dimension with near equivalent descriptive power to the original space.

High dimensionality also poses challenges in interpreting data. Correlations between features, or sets of features become difficult to discover as the number of potential pairings increase. Datasets with more than three dimensions pose additional challenges as visualizing such datasets is difficult and often impractical, rendering visual interpretation methods unusable.

\section{Exploring the chemical space}\label{sec:exploring-the-chemical-space}

One of the more important goals of modern pharmaceutical research is the discovery of novel drug-like molecules. This development however is a long and costly process. The entire procedure can in some cases take 20 years~\cite{bib:rndprod}, and may cost as much as 2.6 billion dollars~\cite{bib:2.6billion}. During the development, thousands of potential compounds are being tested, of which only half a dozen candidates will reach clinical trials, where their effects are tested in humans. In most cases, only one drug is approved so that it can reach doctors and patients.

In order to reduce the time cost of the procedure, as well as the financial cost, targeted search of the chemical space is necessary. This conclusion is also supported by the fact that while scientists have successfully synthesised millions of molecules with molar weight less than 500, the number of potential such molecules may be $10^{24}$, or according to some estimates, even $10^{60}$~\cite{bib:1060}, making random searches of the chemical space effectively impossible.

Using targeted searching on the space of drug-like molecules is a heavily researched topic. Algorithms exist, however, finding efficient searching algorithms without enumerating every element of the subspace is difficult. Deep neural networks have been used for \textit{de novo} molecule generation with promising results~\cite{bib:molsearch}. One particularly promising method is the use of VAE's \textit{variational autoencoders}~\cite{bib:vae} to transform the chemical space into a latent space that is smooth over chemical structure.

The number of dimensions of an autoencoder's latent space is usually between 32 and 128, which is rather large, and carries the challenges in the understanding of the underlying structure presented in section~\ref{sec:motivation}. The analysis of such latent spaces is important for finding novel molecules with desirable drug-like properties.

\section{Unsupervised clustering}\label{sec:unsupervised-clustering}

In data science, a very common task is discovering similarities and dissimilarities between data points. Categorizing points of data is perhaps the most obvious example of this, however, more general tasks exist in datasets where no such clean categories exist. These problems require a more general solution for representing similarities of points.

Cluster analysis is the grouping of objects such that objects in the same cluster are more similar to each other than they are to objects in another cluster. The classification into clusters is done using criteria such as smallest distances, density of data points, graphs, or various statistical distributions. Cluster analysis has  wide applicability, including in unsupervised machine learning, data mining, statistics, graph analytics,  image processing, and numerous physical and social science applications.

Clustering is used to identify groups of similar objects in datasets with two or more variable quantities. In practice, this data may be collected from marketing, biomedical, or geospatial databases, among datasets that come from preprocessing such databases.

Clustering is often used to discover underlying structure in data through \textit{unsupervised learning}. No labels are given to the learning algorithm, leaving it on its own to find structure in its input. Unsupervised learning can be a goal in itself (discovering hidden patterns in data) or a means towards an end (feature learning).

\section{Outline of work}


In my thesis, I examine five dimension reduction algorithms, namely PCA, t-SNE, UMAP, TriMAP and PaCMAP. In chapter (\ref{ch:dimension-reduction-algorithms}), I introduce each algorithm, and provide simple description of how they work. I will also present a basic comparison of based on these descriptions. 

In chapter (\ref{ch:resources}), I talk about the resources that I had during my work. This includes the hardware on which I ran the algorithms and the software packages that I used, including the implementations of all algorithms presented. I also describe the dataset that I used in my investigation.

Following that in chapter (\ref{ch:own-work}), I compare the algorithms in detail, running them with different parametrization on the latent dataset of drug-like molecules. With optimizing parameters and running tests, I examine which unsupervised clustering algorithm works best on the given dataset.

Finally, in chapter (\ref{ch:conclusion}), I summarize my findings and talk about future plans.